
\section{Introduction}

In this section, we introduce a mathematical framework that describes the sports betting market. We model the interactions between bettors and bookmakers, considering key elements such as match outcomes, associated probabilities, odds, and bankroll dynamics. Bettors aim to maximize their utility by placing bets based on their estimated probabilities of match outcomes, while bookmakers strategically set odds to manage risk and ensure profitability. This agent-based framework enables a systematic analysis of decision-making processes in sports betting, balancing risk and reward. The complete list of notations can be found here \ref{appendix:list_of_notations}.

\section{Mathematical Formalization of Sports Betting}

\subsection{Matches and Outcomes}

At any given time \( t \in \mathbb{R}^+ \), we define the set of matches available for betting as:

\[
\mathbb{M}(t) = \{ m^1, m^2, \dots, m^{M(t)} \}
\]

where \( M(t) \in \mathbb{N} \) represents the total number of matches available at time \( t \).

For each match \( m^k \in \mathbb{M}(t) \), there is a set of possible outcomes:

\[
\Omega^k = \{ \omega_1^k, \omega_2^k, \dots, \omega_{N^k}^k \}
\]

where \( N^k \in \mathbb{N} \) represents the number of possible outcomes for match \( m^k \).

\paragraph{Example:} 
In a football match, possible outcomes might be chosen as \{home team wins, draw, away team wins\}, so \( N^k = 3\) \(\forall k\).

\subsection{Probabilities of Outcomes}

We define \( \mathbb{P}_Y( \omega_i^k ) \) as the probability that outcome \( \omega_i^k \) occurs for match \( m^k \), given the state of the world \( Y \) at time \( t \):

\[
r_i^k(t) = \mathbb{P}_Y( \omega_i^k )
\]

These probabilities may change over time as new information becomes available.

We introduce the random variable \( X_i^k \) associated with outcome \( \omega_i^k \):

\[
X_i^k = \begin{cases}
1, & \text{if outcome } \omega_i^k \text{ occurs}, \\
0, & \text{otherwise}.\\
\end{cases}
\]

Therefore, \( r_i^k(t) = \mathbb{P}_Y( X_i^k = 1 ) \). \\

\paragraph{Example:} 
Consider a football match \( m^k \) between Team A and Team B. The possible outcomes \( \omega_i^k \) are:

\[
\omega_1^k = \text{Team A wins}, \quad \omega_2^k = \text{Draw}, \quad \omega_3^k = \text{Team B wins}.
\]

At time \( t \), based on current information \( Y \) (such as form, injuries, and past results), the probabilities of these outcomes are:

\[
r_1^k(t) = \mathbb{P}_Y(\text{Team A wins}), \quad r_2^k(t) = \mathbb{P}_Y(\text{Draw}), \quad r_3^k(t) = \mathbb{P}_Y(\text{Team B wins}).
\]

For example, if \( r_1^k(t) = 0.55 \), it means there is a 55\% chance that Team A will win.

\subsection{Bettors and Bookmakers}

Let \( \mathbb{J} \) be the set of bettors, and \( \mathbb{B} \) be the set of bookmakers.

Each bettor \( J \in \mathbb{J} \) has a bankroll at time \( t \), denoted by:

\[
B_{\text{bettor}}^J(t)
\]

Similarly, each bookmaker \( B \in \mathbb{B} \) has a bankroll at time \( t \), denoted by:

\[
B_{\text{bookmaker}}^B(t)
\]

\subsection{Odds}

At time \( t \), bookmaker \( B \) offers odds on the outcomes of matches. For match \( m^k \), the odds offered by bookmaker \( B \) are:

\[
\mathbb{O}^k(B, t) = \{ o_1^{k,B}(t), o_2^{k,B}(t), \dots, o_{N^k}^{k,B}(t) \}
\]

where \( o_i^{k,B}(t) \) represents the odds offered on outcome \( \omega_i^k \) of match \( m^k \) at time \( t \). \\

\paragraph{Example:}
Consider the same football match \( m^k \) between Team A and Team B. At time \( t \), bookmaker \( B \) offers the following odds:

\[
\mathbb{O}^k(B, t) = \{ 2.00, 3.50, 4.00 \}
\]

Where \( o_1^{k,B}(t) = 2.00 \) for Team A to win, \( o_2^{k,B}(t) = 3.50 \) for a draw, and \( o_3^{k,B}(t) = 4.00 \) for Team B to win.

These odds represent the potential payouts for each outcome.

\subsection{Bets and Wagers}

At time \( t \), bettor \( J \) may choose to place bets on various outcomes. We define:

\begin{itemize}
    \item \( f_i^{k,J}(t) \): The fraction of bettor \( J \)'s bankroll \( B_{\text{bettor}}^J(t) \) that is wagered on outcome \( \omega_i^k \) of match \( m^k \).
    \item \( b_i^{k,J}(t) \): The bookmaker \( B \) with whom bettor \( J \) places the bet on outcome \( \omega_i^k \) of match \( m^k \).
\end{itemize}

Therefore, the amount wagered by bettor \( J \) on outcome \( \omega_i^k \) at time \( t \) is:

\[
w_i^{k,J}(t) = f_i^{k,J}(t) \times B_{\text{bettor}}^J(t)
\]


\paragraph{Example:}
Consider bettor \( J \) with a bankroll of \( B_{\text{bettor}}^J(t) = 100 \) units at time \( t \). Bettor \( J \) decides to wager:

\[
f_1^{k,J}(t) = 0.2 \quad \text{(20\% of the bankroll on Team A to win)}
\]

Thus, the amount wagered is:

\[
w_1^{k,J}(t) = 0.2 \times 100 = 20 \text{ units}
\] 

Bettor \( J \) places the 20-unit bet with bookmaker \( B \).

\subsection{Bankroll Evolution}

The evolution of the bettors' and bookmakers' bankrolls depends on the outcomes of the matches and the settlement of bets.

\subsubsection{Bettor's Bankroll Evolution}

The bankroll of bettor \( J \) at time \( t \) is given by:

\[
B_{\text{bettor}}^J(t) = B_{\text{bettor}}^J(0) + \int_0^t \sum_{b \in \mathcal{B}_{\text{settled}}^J(\tau)} G_{\text{bettor}}^J(b) \, d\tau
\]

where:

\begin{itemize}
    \item \( \mathcal{B}_{\text{settled}}^{J}(s) \) is the set of bets placed by bettor \( J \) that are settled at time \( s \).
    \item \( G_{\text{bettor}}^J(b) \) is the gain or loss from bet \( b \), calculated as:
    \[
    G_{\text{bettor}}^J(b) = w^{J}(b) \times \left( o^{B}(b) \times X(b) - 1 \right)
    \]
    \begin{itemize}
        \item \( w^{J}(b) \) is the amount wagered on bet \( b \).
        \item \( o^{B}(b) \) is the odds offered by bookmaker \( B \) for bet \( b \).
        \item \( X(b) \) indicates whether the bet was successful (\( X(b) = 1 \)) or not (\( X(b) = 0 \)).
    \end{itemize}
\end{itemize}

\paragraph{Example:}

Consider bettor \( J \) starts with a bankroll of \( B_{\text{bettor}}^J(0) = 100 \) units. At time \( t_1 \), the bettor places a bet of \( w^{J}(b) = 20 \) units on a match with odds \( o^{B}(b) = 2.50 \) offered by bookmaker \( B \).

If the outcome \( X(b) = 1 \) (the bettor wins the bet), the gain from the bet is:

\[
G_{\text{bettor}}^J(b) = 20 \times (2.50 \times 1 - 1) = 30 \text{ units}
\]

Thus, the updated bankroll at time \( t_1 \) is:

\[
B_{\text{bettor}}^J(t_1) = 100 + 30 = 130 \text{ units}
\]

If the bettor loses another bet at time \( t_2 \) with a wager of 30 units on odds of 3.00, then \( X(b) = 0 \) and the loss is:

\[
G_{\text{bettor}}^J(b) = 30 \times (3.00 \times 0 - 1) = -30 \text{ units}
\]

The bankroll at time \( t_2 \) becomes:

\[
B_{\text{bettor}}^J(t_2) = 130 - 30 = 100 \text{ units}
\]

\subsubsection{Bookmaker's Bankroll Evolution}

Similarly, the bankroll of bookmaker \( B \) at time \( t \) is given by:

\[
B_{\text{bookmaker}}^B(t) = B_{\text{bookmaker}}^B(0) + \int_0^t \sum_{J \in \mathcal{J}} \sum_{b \in \mathcal{B}_{\text{settled}}^{B,J}(\tau)} G_{\text{bookmaker}}^B(b) \, d\tau
\]

where:

\begin{itemize}

    \item \( \mathcal{J} \) is the set of all bettors \( \{ J_1, J_2, \dots, J_N \} \) placing bets with bookmaker \( B \).
    \item \( \mathcal{B}_{\text{settled}}^{B,J}(s) \) is the set of bets accepted by bookmaker \( B \) from bettor \( J \) that are settled at time \( s \).
    \item \( G_{\text{bookmaker}}^B(b) \) is the gain or loss from bet \( b \), which now takes into account multiple bettors \( J \), calculated as:
    
    \[
    G_{\text{bookmaker}}^B(b) = w^{J}(b) \times \left( 1 - o^{B}(b) \times X(b) \right)
    \]
    
    where:
    \begin{itemize}
        \item \( w^{J}(b) \) is the amount wagered by bettor \( J \) on bet \( b \).
        \item \( o^{B}(b) \) is the odds offered by bookmaker \( B \) for bet \( b \).
        \item \( X(b) \) indicates whether the bet was successful (\( X(b) = 1 \)) or not (\( X(b) = 0 \)).
    \end{itemize}
\end{itemize}
\subsubsection{Impact of Multiple Bettors}

For each bet \( b \), the gain or loss for bookmaker \( B \) depends on which bettor placed the bet. If bettor \( J \) wins, bookmaker \( B \) pays out, and if bettor \( J \) loses, bookmaker \( B \) gains:

\[
G_{\text{bookmaker}}^B(b) = - G_{\text{bettor}}^J(b)
\]

Thus, for each bet placed by a bettor \( J \), the bookmaker’s gain is equal to the bettor’s loss, and vice versa. With multiple bettors, the bookmaker's bankroll reflects the combined gains and losses from all bets settled across the bettors \( J_1, J_2, \dots, J_N \).


\subsection{Bankroll Factor}

To abstract from the initial bankroll amounts, we can define the \textit{Bankroll Factor} for bettors and bookmakers.

\subsubsection{Bettor's Bankroll Factor}

The bankroll factor for bettor \( J \) at time \( t \) is defined as:

\[
BF_{\text{bettor}}^J(t) = \frac{B_{\text{bettor}}^J(t)}{B_{\text{bettor}}^J(0)}
\]

This represents the growth of the bettor's bankroll relative to their initial bankroll.

\subsubsection{Bookmaker's Bankroll Factor}

Similarly, the bankroll factor for bookmaker \( B \) at time \( t \) is:

\[
BF_{\text{bookmaker}}^B(t) = \frac{B_{\text{bookmaker}}^B(t)}{B_{\text{bookmaker}}^B(0)}
\]

\subsection{Gain Calculation}

The cumulative gain for bettor \( J \) up to time \( t \) is:

\[
G_{\text{bettor}}^J(t) = B_{\text{bettor}}^J(t) - B_{\text{bettor}}^J(0) = B_{\text{bettor}}^J(0) \left( BF_{\text{bettor}}^J(t) - 1 \right)
\]

Similarly, for bookmaker \( B \):

\[
G_{\text{bookmaker}}^B(t) = B_{\text{bookmaker}}^B(t) - B_{\text{bookmaker}}^B(0) = B_{\text{bookmaker}}^B(0) \left( BF_{\text{bookmaker}}^B(t) - 1 \right)
\]


\subsection{Utility Function}

The utility function \( U \) represents the agent's preferences regarding risk and reward, crucial in decision-making under uncertainty \cite{KahnemanTversky1979}. Bettors and bookmakers use this function to optimize their gains over time while minimizing risk. Unlike expected returns, utility functions incorporate risk preferences, allowing agents to balance the trade-off between potential gains and variability 
\cite{Markowitz1952} \cite{Arrow1971} \cite{Pratt1964}.

\subsubsection{Forms of Utility Functions}

Different utility functions capture varying risk attitudes, ranging from risk-neutral to risk-averse behaviors. Below are the common types of utility functions in the betting market:

\paragraph{1. Expected Value Utility (Risk-Neutral)}

The simplest form, where utility is directly proportional to wealth:

\[
U(B) = B
\]

Agents using this function are risk-neutral, focusing solely on maximizing expected returns without considering risk.

\paragraph{2. Logarithmic Utility (Moderate Risk Aversion)}

Logarithmic utility models constant relative risk aversion (CRRA) and is expressed as:

\[
U(B) = \ln(B)
\]

This function reflects diminishing marginal utility of wealth, balancing risk and reward, commonly used in the Kelly Criterion \cite{Kelly1956} \cite{Thorp1975} for long-term growth.

\paragraph{3. Power Utility (CRRA)}

A generalization of logarithmic utility, with risk aversion controlled by \( \gamma \):

\[
U(B) = \frac{B^{1 - \gamma}}{1 - \gamma}, \quad \gamma \neq 1
\]

Higher \( \gamma \) values indicate greater risk aversion. When \( \gamma = 1 \), the function becomes logarithmic.

\paragraph{4. Exponential Utility (Constant Absolute Risk Aversion - CARA)}

The exponential utility models constant absolute risk aversion (CARA):

\[
U(B) = -e^{-\alpha B}
\]

Here, \( \alpha \) controls risk aversion. Agents using this function maintain consistent risk preferences regardless of wealth level.

\paragraph{5. Quadratic Utility}

Quadratic utility is given by:

\[
U(B) = B - \frac{\lambda}{2} B^2
\]

Though it captures increasing risk aversion, it has the drawback of implying decreasing utility at higher wealth levels, making it less commonly used.

\subsubsection{Implications of Different Utility Functions}

Each utility function models specific risk preferences, influencing the agent’s decisions:

\paragraph{Risk-Neutral Behavior}

Agents with linear utility (\( U(B) = B \)) focus solely on maximizing returns, indifferent to risk. This behavior is rare in practice due to the inherent risks in betting.

\paragraph{Risk-Averse Behavior}

Utility functions like logarithmic, power, and exponential represent risk-averse behavior:

\begin{itemize}
    \item \textbf{Logarithmic Utility:} Moderate risk aversion, favoring long-term growth.
    \item \textbf{Power Utility (CRRA):} Flexibility in modeling different degrees of risk aversion via \( \gamma \).
    \item \textbf{Exponential Utility (CARA):} Constant risk aversion regardless of wealth.
\end{itemize}

\paragraph{Risk-Seeking Behavior}

Agents may occasionally exhibit risk-seeking behavior, favoring higher variance. This is typically modeled by utility functions with convex regions or negative coefficients but is unsustainable in the long term.

\subsubsection{Choosing an Appropriate Utility Function}

Selecting the right utility function depends on:

\begin{itemize}
    \item \textbf{Risk Preference:} It should reflect the agent’s risk tolerance.
    \item \textbf{Mathematical Tractability:} Functions like logarithmic utility offer simpler analytical solutions.
    \item \textbf{Realism:} The chosen function should realistically model the agent’s behavior in the market.
\end{itemize}


\section{General Agent-Based Betting Framework}

In order to model the decision-making processes of bettors and bookmakers in sports betting, we adopt a general agent-based framework \cite{Ferguson1967}. This framework allows us to formalize the interactions between agents (bettors and bookmakers) and the environment (the sports betting market) in a comprehensive and systematic manner. By defining the state space, action space, and other essential components in the most general terms, we can capture the complexity of sports betting and lay the groundwork for more specific analyses.

\subsection{Agents in the Betting Market}

There are two primary types of agents in the sports betting market:

\begin{itemize}
    \item \textbf{Bettors (Players):} Individuals or entities who place bets on the outcomes of sporting events with the aim of maximizing their returns.
    \item \textbf{Bookmakers:} Organizations or individuals who offer betting opportunities by setting odds on the possible outcomes of sporting events, aiming to maximize their profits.
\end{itemize}

Each agent operates based on their own objectives, information, and strategies, interacting with the environment and other agents through their actions.

\subsection{State Space}

At any given time \( t \in \mathbb{R}^+ \), the state of the sports betting environment, denoted by \( S(t) \), encompasses all the information relevant to the agents' decision-making processes. The state space \( \mathcal{S} \) is the set of all possible states \( S(t) \).

The state \( S(t) \) can be defined as:

\[
S(t) = \left( \mathbb{M}(t), \Omega(t), \mathbb{O}(t), B_{\text{bettor}}(t), B_{\text{bookmaker}}(t), H(t), \mathcal{I}(t) \right)
\]

where:

\begin{itemize}
    \item \( \mathbb{M}(t) \): The set of all matches available at time \( t \).
    \item \( \Omega(t) \): The set of possible outcomes for each match in \( \mathbb{M}(t) \).
    \item \( \mathbb{O}(t) \): The set of odds offered by bookmakers for each possible outcome at time \( t \).
    \item \( B_{\text{bettor}}(t) \): The set of bettors' bankrolls at time \( t \).
    \item \( B_{\text{bookmaker}}(t) \): The set of bookmakers' bankrolls at time \( t \).
    \item \( H(t) \): The history of past events up to time \( t \), including past bets, match results, and odds movements.
    \item \( \mathcal{I}(t) \): Any additional information available to the agents at time \( t \), such as team news, player injuries, weather conditions, etc.
\end{itemize}

The state \( S(t) \) encapsulates all the variables that can influence the agents' decisions, making it comprehensive and general.

\subsection{Action Space}

At each time \( t \), agents choose actions from their respective action spaces:

\subsubsection{Bettors' Action Space}

The action space for a bettor \( J \) at time \( t \), denoted by \( \mathcal{A}_{\text{bettor}}^J(t) \), consists of all possible betting decisions they can make. An action \( A_{\text{bettor}}^J(t) \in \mathcal{A}_{\text{bettor}}^J(t) \) can be defined as:

\[
A_{\text{bettor}}^J(t) = \left\{ \left( f_i^k \right) \mid f_i^k \in [0,1] , \sum_{i,k}f_i^k <= 1\right\}
\]

where:

\begin{itemize}
    \item \( f_i^k \): The fraction of the bettor's bankroll \( B_{\text{bettor}}^J(t) \) to wager on outcome \( \omega_i^k \).
\end{itemize}

Hence, the bettor chose the outcomes to bet on by assigning 0 (no bet) or more to an outcome at a given time \(t\).


\subsubsection{Bookmakers' Action Space}

The action space for a bookmaker \( B \) at time \( t \), denoted by \( \mathcal{A}_{\text{bookmaker}}^B(t) \), can be simplified to the selection of odds for each outcome. An action \( A_{\text{bookmaker}}^B(t) \in \mathcal{A}_{\text{bookmaker}}^B(t) \) is defined as:

\[
A_{\text{bookmaker}}^B(t) = \left\{ \mathbb{O}^k(B, t) = \{ o_i^k \mid o_i^k \in [1, \infty) \right\}
\]

where:

\begin{itemize}
    \item \( o_i^k \): The odds set by the bookmaker \( B \) for outcome \( \omega_i^k \) of match \( m^k \) at time \( t \). 
\end{itemize}

If \( o_i^k = 1 \), the bookmaker does not offer bets on outcome \( \omega_i^k \). If all odds \( o_i^k = 1 \) for a match \( m^k \), the bookmaker does not offer that match for betting.

\paragraph{Example:}
At time \( t \), bettor \( J \) allocates fractions of their \( 100 \) unit bankroll across two matches, with three possible outcomes:

\[
f = \begin{pmatrix}
0.3 & 0.2 & 0 \\
0.5 & 0 & 0 \\
\end{pmatrix}
\]

The bookmaker sets the following odds for each outcome:

\[
o = \begin{pmatrix}
2.50 & 3.00 & 4.00 \\
1.80 & 2.90 & 3.50 \\
\end{pmatrix}
\]

This means bettor \( J \) wagers 30 units on \( \omega_1^1 \) (Team A wins \( m^1 \)), 20 units on \( \omega_2^1 \) (draw in \( m^1 \)), and 50 units on \( \omega_1^2 \) (Team A wins \( m^2 \)).


\subsection{Transition Dynamics}

The state transitions \( \frac{dS(t)}{dt} \) are governed by the interactions between the agents' actions and the environment. The transition dynamics can be described in general terms:

\[
\frac{dS(t)}{dt} = \Phi\left( S(t), A_{\text{bettor}}(t), A_{\text{bookmaker}}(t), \epsilon(t) \right)
\]

where:

\begin{itemize}
    \item \( \Phi \) is the state transition function.
    \item \( A_{\text{bettor}}(t) \): The set of all bettors' actions at time \( t \).
    \item \( A_{\text{bookmaker}}(t) \): The set of all bookmakers' actions at time \( t \).
    \item \( \epsilon(t) \): Represents the stochastic elements inherent in sports outcomes and market dynamics, modeled as random variables.
\end{itemize}

The transition function \( \Phi \) captures how the state evolves due to:

\begin{itemize}
    \item The resolution of matches (outcomes becoming known), represented by changes in outcome variables over time..
    \item The settlement of bets (adjustment of bettors' and bookmakers' bankrolls).
    \item Changes in available matches and odds for the next time period.
    \item Updates to the history \( H(t) \) and information set \( \mathcal{I}(t) \), represented by \(\frac{dH(t)}{dt}\) and \(\frac{d\mathcal{I}(t)}{dt}\).
\end{itemize}


\subsection{Policies}

Each agent follows a policy that guides their decision-making process:

\subsubsection{Bettors' Policy}

A bettor's policy \( \pi_{\text{bettor}}^J \) is a mapping from states to actions:

\[
\pi_{\text{bettor}}^J: \mathcal{S} \rightarrow \mathcal{A}_{\text{bettor}}^J
\]

The policy determines how the bettor decides on which bets to place and how much to wager, based on the current state \( S(t) \).

\subsubsection{Bookmakers' Policy}

A bookmaker's policy \( \pi_{\text{bookmaker}}^B \) is a mapping from states to actions:

\[
\pi_{\text{bookmaker}}^B: \mathcal{S} \rightarrow \mathcal{A}_{\text{bookmaker}}^B
\]

The policy dictates how the bookmaker sets odds and offers betting opportunities, considering factors like market demand, risk management, and competitive positioning.

\subsection{Objectives and Constraints}

Each agent aims to optimize an objective function over time, such as maximizing expected utility or profit, subject to specific constraints that reflect their operational limitations and risk management considerations.

\subsubsection{Bettors' Objective}

The bettor seeks to maximize a chosen utility over a time horizon \( T \):

\[
\max_{\pi_{\text{bettor}}^J} \quad \mathbb{E} \left[ U^{J} \left( BF_{\text{bettor}}^J(T) \right) \right]
\]

\subsubsection{Constraints for the Bettor}

The bettor's optimization problem is subject to the following mathematical constraints: 

\begin{itemize}
    \item 1. Budget Constraint at Each Time \( t \):

   The total fraction of the bankroll wagered on all outcomes cannot exceed 1 at any time \( t \):

   \[
   \sum_{k=1}^{M(t)} \sum_{i=1}^{N^k} f_i^{k,J}(t) \leq 1 \quad \forall t
   \]

   where:
   \begin{itemize}
       \item \( f_i^{k,J}(t) \) is the fraction of the bettor \( J \)'s bankroll \( BF_{\text{bettor}}^J(t) \) wagered on outcome \( i \) of match \( k \) at time \( t \).
       \item \( M(t) \) is the total number of matches available at time \( t \).
       \item \( N^k \) is the number of possible outcomes for each match \(k\).
   \end{itemize}


\item 2. Non-Negativity of Wager Fractions:

   The bettor cannot wager negative fractions of the bankroll:

   \[
   f_i^{k,J}(t) \geq 0 \quad \forall i, k, t
   \]
\end{itemize}



\subsubsection{Bookmakers' Objective}

The bookmaker aims to maximize a chosen utility over a time horizon \( T \):

\[
\max_{\pi_{\text{bookmaker}}^B} \quad \mathbb{E} \left[ U^{B} \left( BF_{\text{bookmaker}}^B(T) \right) \right]
\]


\subsubsection{Constraints for the Bookmaker}

The bookmaker's optimization problem is subject to the following mathematical constraints:

\begin{itemize}
    \item 1. Liquidity Constraint:
    
       The bookmaker must ensure sufficient funds to cover potential payouts:
    
       \[
       BF_{\text{bookmaker}}^B(t) \geq \text{Maximum Potential Liability at } t
       \]
    
       This ensures that the bookmaker's bankroll at time \( t \) is greater than or equal to the maximum possible payout based on the accepted bets.
    
    \item 2. Odds Setting Constraints:
    
       The odds must be set to ensure profitability and competitiveness:
    
       \begin{itemize}
          \item Overround Constraint (Bookmaker's Margin):
    
            For each match \( k \), the sum of the implied probabilities must exceed 1:
    
            \[
            \sum_{i=1}^{N^k} \frac{1}{o_i^k(t)} = 1 + \epsilon^k(t) \quad \forall k, t
            \]
    
            Here, \( \epsilon^k(t) > 0 \) represents the bookmaker's margin for match \( k \) at time \( t \).
    
          \item Margin Bound:
    
            To balance profitability and competitiveness, we impose the following bound on \( \epsilon^k(t) \):
    
            \[
            \epsilon_{\text{min}} \leq \epsilon^k(t) \leq \epsilon_{\text{max}} \quad \forall k, t
            \]
    
            This ensures that the margin \( \epsilon^k(t) \) stays within a specified range, keeping the odds competitive enough to attract bettors while securing a minimum margin for profitability.
          
          \item Competitive Odds Constraint:
    
            The odds \( o_i^k(t) \) must remain competitive, influenced by market averages or competitors' odds. Therefore, the bookmaker may aim to keep \( \epsilon^k(t) \) as low as possible while maintaining profitability and covering risk.
       \end{itemize}
\end{itemize}




\section{Reduction to the Studied Case}

Building upon the general agent-based betting framework, we aim to simplify the agent-based betting framework and reduce computational complexity. We transition from a dynamic to a static optimization model by introducing key assumptions. By assuming immediate resolution of bets and the absence of intertemporal dependencies—where current decisions do not influence future opportunities—we make the static and dynamic problems effectively equivalent for our purposes. This simplification allows us to optimize agents' decisions at each time step independently, facilitating the derivation of optimal solutions without the need for complex dynamic programming. However, this reduction comes at a cost, notably in terms of long-term interpretability, as the model no longer accounts for cumulative effects and evolving dynamics over time.

\subsection{Hypotheses for the Constrained Problem}

\begin{enumerate}
    \item \textbf{No Intertemporal Dependencies (Additive Utility Function):} 
    Utility is additive over time, meaning decisions at time \( t \) do not affect future periods. The agent maximizes utility independently at each step, simplifying the problem into sequential sub-problems.
    
    \textit{Reason:} This eliminates the need to account for future wealth in current decisions, reducing complexity.

    \item \textbf{Discrete Time Steps:} 
    Time is divided into discrete intervals where decisions are made periodically. Bets are resolved by the end of each period before moving to the next. \( t = 0, 1, 2, \dots, T \)
    
    \textit{Reason:} Discrete time steps reduce the dynamic problem to a series of static decisions, simplifying optimization.

    \item \textbf{Non-Overlapping Bets:} 
    Bets are settled within the same period, ensuring that wealth at the end of each period is fully available for the next, avoiding unresolved wagers impacting future decisions.
    
    \textit{Reason:} This ensures no carryover of unresolved bets, keeping each period's wealth independent.

    \item \textbf{Independence of Match Outcomes:} 
    Match outcomes are independent random events, meaning there is no correlation between the results of different matches.
    
    \textit{Reason:} This simplifies probability calculations by eliminating the need to model inter-match dependencies.

    \item \textbf{Static Information Environment:} 
    Information is fixed within each period. No new data arrives mid-period, and updates are considered only in the next time step.
    
    \textit{Reason:} A static environment avoids real-time strategy adjustments, making the problem more manageable.
\end{enumerate}

These assumptions significantly simplify the model by reducing the complexity inherent in a dynamic optimization problem, but they also modify or limit certain long-term interpretations, such as how future wealth or intertemporal risk is managed across multiple betting periods.

\subsection{Simplification of the Utility Maximization Problem}

With no overlapping bets and a static information environment, agents do not need to consider how current actions might affect future opportunities or states. This myopic decision-making approach allows agents to focus solely on the current time period, simplifying their optimization problem to a static one.

Hence, the agents' objective functions depend only on the current wealth and the outcomes of bets placed in the current period. The expected utility maximization problem at each time \( t \) becomes:

For bettors:
\[
\max_{\{ f_i^{k,J}(t) \}} \quad \mathbb{E} \left[ U\left( B_{\text{bettor}}^J(t+1) \right) \mid S(t) \right]
\]

For bookmakers:
\[
\max_{\{ o_i^k(t) \}} \quad \mathbb{E} \left[ U\left( B_{\text{bookmaker}}(t+1) \right) \mid S(t) \right]
\]

where \( S(t) \) is the state at time \( t \), which includes the available matches, odds, and the agents' current bankrolls.

\subsection{Dynamic and Total Utility under Assumptions}

In our framework, under the assumption of discrete time steps and no intertemporal dependencies, the total utility across all periods \( T \) is given by the sum of the static utilities at each time step:

\[
U_{\text{total}} = \sum_{t=1}^{T} U(B(t)).
\]

This assumes that decisions are made independently at each \( t \), with the utility depending solely on the wealth \( B(t) \) at that moment. Additive utility functions, such as \( U(B) = B \), respect this assumption directly, meaning maximizing the utility at each step also maximizes total utility.

However, logarithmic and exponential utilities do not preserve a simple additive structure due to risk preferences that influence future decisions. While linear utility maintains additivity, \( U(B) = \ln(B) \) and \( U(B) = -e^{-\alpha B} \) do not.

\subsubsection{Utility Functions Respecting Additivity}
\begin{itemize}
\item Linear utility: \( U(B) = B \)
\end{itemize}

\subsubsection{Utility Functions Not Respecting Additivity}
\begin{itemize}
\item Logarithmic utility: \( U(B) = \ln(B) \)
\item Exponential utility: \( U(B) = -e^{-\alpha B} \)
\item CRRA: \(U(B) = \frac{B^{1 - \gamma}}{1 - \gamma}, \quad \gamma \neq 1\)
\item Quadratic utility: \(U(B) = B - \frac{\lambda}{2} B^2\)
\end{itemize}

\subsubsection{Approximation with Additive Properties}
By using a first-order Taylor expansion for \( \ln(B) \) or \( -e^{-\alpha B} \), these utilities can become approximately additive. For small deviations around \( B \), we approximate:

\[
\ln(B) \approx \ln(B_0) + \frac{B - B_0}{B_0}, \quad -e^{-\alpha B} \approx -e^{-\alpha B_0} + \alpha e^{-\alpha B_0} (B - B_0)
\]

These approximations are linear in \( B \), making the utility functions additive for small changes in wealth. Under these assumptions, the complexity of the problem is reduced, allowing the use of simpler optimization techniques without fully abandoning the original utility structure.

\subsubsection{Non-Additive Utility Maximization and Long-Term Interpretation}

When maximizing non-additive utility functions (such as logarithmic or exponential) at each step \( t \), the interpretation of utility over the entire period \( T \) changes. Unlike additive functions, where the total utility is simply the sum of the utilities at each time step, non-additive functions induce a more complex relationship between short-term and long-term behavior.

For non-additive utilities, maximizing utility at each step does not guarantee maximization of the utility across the entire period. The decisions made at each step can interact non-linearly across time, meaning that the long-term growth or risk profile may differ significantly from the one-step behavior. This highlights the difference between local (step-by-step) optimization and the global impact over the entire period.

\subsubsection{Interpretation of Log Utility in Terms of Long-Term Geometric Growth}

Maximizing the logarithmic utility at each time step involves maximizing the expected utility:

\[
\max_{f(t)} \mathbb{E}\left[ \ln B_{\text{agent}}(t+1) \, \big| \, \mathcal{F}_t \right],
\]

where \( B_{\text{agent}}(t+1) \) is the wealth at time \( t+1 \), \( f(t) \) represents the decision variables at time \( t \), and \( \mathcal{F}_t \) denotes the information available at time \( t \).

The total utility over \( T \) periods is given by:

\[
U_{\text{total}} = \sum_{t=1}^{T} \ln B_{\text{agent}}(t) = \ln\left( \prod_{t=1}^{T} B_{\text{agent}}(t) \right).
\]

Taking the expectation of the total utility, we have:

\[
\mathbb{E}[ U_{\text{total}} ] = \mathbb{E}\left[ \ln\left( \prod_{t=1}^{T} B_{\text{agent}}(t) \right) \right].
\]

However, due to the concavity of the logarithm and the properties of expectations, we cannot simplify this expression to \( \ln \left( \prod_{t=1}^{T} \mathbb{E}[ B_{\text{agent}}(t) ] \right) \) unless the \( B_{\text{agent}}(t) \) are deterministic. The expected value of the logarithm of a product of random variables is not equal to the logarithm of the product of their expectations.

To interpret \( \mathbb{E}[ U_{\text{total}} ] \) in terms of expected wealth and variance, we can use a second-order Taylor expansion of the logarithm around \( \mathbb{E}[ B_{\text{agent}}(t) ] \):

\[
\mathbb{E}[ \ln B_{\text{agent}}(t) ] \approx \ln \mathbb{E}[ B_{\text{agent}}(t) ] - \frac{1}{2} \frac{ \mathbb{V}\mathrm{ar}[ B_{\text{agent}}(t) ] }{ \left( \mathbb{E}[ B_{\text{agent}}(t) ] \right)^2 }.
\]

Summing over \( T \) periods, we obtain:

\[
\mathbb{E}[ U_{\text{total}} ] \approx \sum_{t=1}^{T} \left( \ln \mathbb{E}[ B_{\text{agent}}(t) ] - \frac{1}{2} \frac{ \mathbb{V}\mathrm{ar}[ B_{\text{agent}}(t) ] }{ \left( \mathbb{E}[ B_{\text{agent}}(t) ] \right)^2 } \right).
\]

This approximation shows that the expected total utility depends on both the expected wealth and the variance at each time step. The logarithmic utility function captures the trade-off between expected wealth growth and risk (variance), penalizing volatility and favoring steady growth.

Over the long term, maximizing the expected logarithmic utility leads to maximizing the \textbf{expected logarithm of cumulative wealth}, which corresponds to maximizing the \textbf{geometric mean return}. This strategy ensures that wealth grows at the highest possible geometric rate, accounting for both returns and risks.

\subsubsection{Long-Term Interpretation of Exponential Utility}

For the exponential utility function \( U(B) = -e^{ -\alpha B } \), where \( \alpha > 0 \) is the coefficient of absolute risk aversion, the total utility over \( T \) periods is:

\[
U_{\text{total}} = \sum_{t=1}^{T} U( B(t) ) = -\sum_{t=1}^{T} e^{ -\alpha B(t) }.
\]

Taking the expectation, we have:

\[
\mathbb{E}[ U_{\text{total}} ] = -\sum_{t=1}^{T} \mathbb{E}\left[ e^{ -\alpha B(t) } \right ].
\]

We cannot simplify \( \mathbb{E}\left[ e^{ -\alpha B(t) } \right ] \) without specifying the distribution of \( B(t) \). However, using a second-order Taylor expansion around \( \mathbb{E}[ B(t) ] \):

\[
\mathbb{E}\left[ e^{ -\alpha B(t) } \right ] \approx e^{ -\alpha \mathbb{E}[ B(t) ] } \left( 1 + \frac{ \alpha^2 }{2} \mathbb{V}\mathrm{ar}[ B(t) ] \right).
\]

Therefore, the expected total utility becomes:

\[
\mathbb{E}[ U_{\text{total}} ] \approx -\sum_{t=1}^{T} e^{ -\alpha \mathbb{E}[ B(t) ] } \left( 1 + \frac{ \alpha^2 }{2} \mathbb{V}\mathrm{ar}[ B(t) ] \right).
\]

This expression highlights that the expected utility depends heavily on both the expected wealth and the variance. As \( \alpha \) increases, the variance term becomes more significant, reinforcing the agent's aversion to risk. The exponential utility function thus focuses on \textbf{risk minimization} and \textbf{capital preservation} over wealth maximization.

\subsubsection{Long-Term Interpretation of Mean-Variance Utility}

For the mean-variance utility, which can be associated with a quadratic utility function \( U(B) = B - \frac{ \lambda }{ 2 } B^2 \) for small variations in \( B \), the expected utility at each time step is:

\[
\mathbb{E}[ U(B(t)) ] = \mathbb{E}[ B(t) ] - \frac{ \lambda }{ 2 } \mathbb{E}[ B(t)^2 ].
\]

Assuming that \( \mathbb{E}[ B(t)^2 ] = \left( \mathbb{E}[ B(t) ] \right)^2 + \mathbb{V}\mathrm{ar}[ B(t) ] \), we have:

\[
\mathbb{E}[ U(B(t)) ] = \mathbb{E}[ B(t) ] - \frac{ \lambda }{ 2 } \left( \left( \mathbb{E}[ B(t) ] \right)^2 + \mathbb{V}\mathrm{ar}[ B(t) ] \right).
\]

Over \( T \) periods, the expected total utility is:

\[
\mathbb{E}[ U_{\text{total}} ] = \sum_{t=1}^{T} \mathbb{E}[ U(B(t)) ].
\]

Simplifying, we obtain:

\[
\mathbb{E}[ U_{\text{total}} ] = \sum_{t=1}^{T} \left( \mathbb{E}[ B(t) ] - \frac{ \lambda }{ 2 } \left( \left( \mathbb{E}[ B(t) ] \right)^2 + \mathbb{V}\mathrm{ar}[ B(t) ] \right) \right).
\]

This expression demonstrates that the agent considers both the expected wealth and the variance, with the parameter \( \lambda \) controlling the trade-off between maximizing returns and minimizing risk.

\subsection{Simplification of State Transitions}

The agents' state variables, particularly their bankrolls, evolve in a straightforward manner without considering future uncertainties or pending bets. The bankroll update equations become:

\[
B_{\text{bettor}}(t+1) = B_{\text{bettor}}(t) + G_{\text{bettor}}(t)
\]

\[
B_{\text{bookmaker}}(t+1) = B_{\text{bookmaker}}(t) + G_{\text{bookmaker}}(t)
\]

where \( G_{\text{bettor}}(t) \) and \( G_{\text{bookmaker}}(t) \) represent the gains or losses realized from bets placed and settled within time \( t \).


\subsection{Detailed Simplification of the Bookmaker's Problem}

Similarly, the bookmaker's optimization problem simplifies under the assumptions:

\paragraph{Objective Function:}

\[
\max_{\{ o_i^k(t) \}} \quad U_{\text{bookmaker}}(t) = \mathbb{E} \left[ U\left( B_{\text{bookmaker}}(t) + G_{\text{bookmaker}}(t) \right) \mid S(t) \right]
\]

\paragraph{Constraints:}

   \[
   BF_{\text{bookmaker}}^B(t) \geq \text{Maximum Potential Liability at } t
   \]

    \[
     \sum_{i=1}^{I} \frac{1}{o_i^k(t)} = 1 + \epsilon^k(t) \quad \text{for all } k, t
    \]


\paragraph{Variables:}

\begin{itemize}
    \item \( o_i^k(t) \): Odds set for outcome \( i \) of match \( k \) at time \( t \).
    \item \( G_{\text{bookmaker}}(t) \): Gain or loss from bets, calculated based on the total bets received and payouts made in the current period.
    \item \(\epsilon^k(t)\): Margin for each match at every time step that the bookmaker set to maximise attractiveness, minimize risque and maximize pay off.
\end{itemize}

\subsection{Reasons for the Simplifications}

We introduce these simplifications for several important reasons:

\subsubsection{Reducing Computational Complexity}

Dynamic optimization problems, especially those involving stochastic elements and intertemporal dependencies, can be highly complex and computationally intensive. By simplifying the problem to a static one, we make it more tractable and amenable to analytical or numerical solutions.

\subsubsection{Simplifying the Use of Historical Odds}

Solving the general dynamic optimization problem requires a sufficiently large history of odds at each time step \( t \) to ensure convergence towards an optimal solution. This includes tracking all relevant historical data for each time step and state \( S \). By reducing the problem to a static case, the need for such an extensive history is eliminated, as the model only relies on current odds. This simplification significantly reduces computational complexity while maintaining the core of the decision-making process.



\subsubsection{Facilitating Analytical Derivations}

With the assumptions of immediate bet resolution and independence, we can derive closed-form solutions or straightforward algorithms for optimal betting strategies, such as the Kelly Criterion for bettors using logarithmic utility functions.

\subsubsection{Focusing on Core Decision-Making Principles}

The simplifications allow us to isolate and analyze the fundamental principles of optimal betting and odds setting without the confounding effects of dynamic interactions. This clarity helps in understanding the key factors that influence agents' decisions in the sports betting market.

\subsection{Limitations of the Simplified Model}

While the simplifications make the model more manageable, they also introduce limitations that should be acknowledged:

\begin{enumerate}
    \item \textbf{Hypothesis: Additive Utility Function with No Intertemporal Dependencies}
        \begin{itemize}
            \item \textbf{Domain of Validity:} Valid when agents focus solely on immediate wealth without concern for future utility.
            \item \textbf{Limitation with Reality:} Agents usually consider future wealth and utility; this assumption ignores long-term planning and risk preferences.
            \item \textbf{Risk:} Ignoring intertemporal effects may result in strategies that maximize short-term gains at the expense of long-term wealth, increasing the risk of ruin or failing to achieve overall financial objectives. Among the utility functions described, only \(U(B)=B\) is additive with time.
        \end{itemize}
        
    \item \textbf{Hypothesis: Discrete Time Steps}
        \begin{itemize}
            \item \textbf{Domain of Validity:} Applicable when betting decisions are made at fixed, regular intervals.
            \item \textbf{Limitation with Reality:} Real betting markets operate continuously; opportunities and information arise at any time, making this assumption somewhat unrealistic.
            \item \textbf{Risk:} By assuming discrete time steps, we risk missing profitable opportunities that occur between intervals and fail to capture the continuous dynamics of the market, leading to suboptimal strategies.
        \end{itemize}

    \item \textbf{Hypothesis: Non-Overlapping Time Steps}
        \begin{itemize}
            \item \textbf{Domain of Validity:} Valid when all bets are short-term and resolved within the same period.
            \item \textbf{Limitation with Reality:} In practice, many bets span multiple periods, and unresolved bets can impact future wealth and decisions; this assumption is restrictive.
            \item \textbf{Risk:} Ignoring overlapping bets may lead to underestimating risk exposure and mismanaging bankrolls, potentially resulting in unexpected losses or liquidity issues.
        \end{itemize}

    \item \textbf{Hypothesis: Independence of Match Outcomes}
        \begin{itemize}
            \item \textbf{Domain of Validity:} Appropriate when matches are truly independent events without any influence on each other.
            \item \textbf{Limitation with Reality:} In reality, match outcomes can be correlated due to common factors; this simplification overlooks potential dependencies.
            \item \textbf{Risk:} Assuming independence when correlations exist can lead to inaccurate probability assessments and risk underestimation, possibly causing overbetting on correlated outcomes and increasing the chance of significant losses.
        \end{itemize}

    \item \textbf{Hypothesis: Static Information Environment}
        \begin{itemize}
            \item \textbf{Domain of Validity:} Suitable for very short periods where no new information is expected to arrive.
            \item \textbf{Limitation with Reality:} Information flows continuously in real markets; ignoring new information is unrealistic and limits strategic adjustments.
            \item \textbf{Risk:} By not accounting for new information, we risk making decisions based on outdated data, leading to poor betting choices and missed opportunities to adjust strategies in response to market changes.
        \end{itemize}

\end{enumerate}

\subsection{Conclusion}

By adhering to the constraints imposed by these hypotheses, we effectively narrow the search space, making it easier to find an optimal solution for our simplified problem. However, it's important to note that the first hypothesis —assuming an additive utility function with no intertemporal dependencies— will not be applied (in every case) in our model. As a result, the optimal solution we derive will differ -if using a non additive utility- from the true optimal solution for the general (using the same utility function), constrained problem under the four next assumptions.



\section{Incorporating Estimated Probabilities in Betting Strategies}

In the real-world sports betting environment, both bettors and bookmakers do not have access to the true probabilities of match outcomes. Instead, they rely on their own estimations based on available information, statistical models, expert opinions, and other predictive tools. These estimated probabilities often differ from the true underlying probabilities and can vary between bettors and bookmakers due to differences in information, analysis techniques, and biases.

This section introduces the concept of estimated probabilities for match outcomes as perceived by bettors and bookmakers, explains the necessity of considering these estimates in modeling betting strategies, and provides analytical derivations for expected gain and variance incorporating these estimated probabilities. We also explore how these differences influence optimal betting strategies, particularly through the application of the Kelly Criterion.

\subsection{Estimated Probabilities}

Let:

\begin{itemize}
    \item \( r_i^k \): The true probability of outcome \( \omega_i^k \) occurring in match \( m^k \).
    \item \( p_i^{k,J} \): The probability estimate of outcome \( \omega_i^k \) as perceived by bettor \( J \).
    \item \( p_i^{k,B} \): The probability estimate of outcome \( \omega_i^k \) as perceived by bookmaker \( B \).
\end{itemize}

Due to the inherent uncertainty and complexity of predicting sports outcomes, the estimated probabilities \( p_i^{k,J} \) and \( p_i^{k,B} \) generally differ from the true probabilities \( r_i^k \) and from each other. These discrepancies are critical in the betting market because they create opportunities for bettors to find value bets (situations where they believe the bookmaker's odds underestimate the true likelihood of an outcome) and for bookmakers to manage their risk and profit margins.


\subsection{Utility Maximization and the Role of Estimated Probabilities}

The bettor aims to maximize their expected utility, which is influenced by both the expected value and the variance of the bankroll factor. The utility function \( U \) encapsulates the bettor's risk preferences.

\subsubsection{Expected Utility in Terms of Bankroll Factor}

The expected utility at time \( t+1 \) is given by:

\[
\mathbb{E}_{p^{J}}\left[ U\left( BF_{\text{bettor}}(t+1) \right) \right] = \sum_{\text{all outcomes}} U\left( BF_{\text{bettor}}(t+1) \right) \times \text{Probability of outcomes}
\]

To compute this expectation, the bettor must consider all possible combinations of match outcomes, weighted by their estimated probabilities \( p_{i_k}^{k,J} \). This requires:

\begin{itemize}
    \item Knowledge of \( p_{i_k}^{k,J} \) for each outcome \( i_k \) in match \( k \).
    \item Calculation of \( BF_{\text{bettor}}(t+1) \) for each possible combination of outcomes.
\end{itemize}

Assuming there are \( M \) matches at time \(t\) to bet on, each with \( N(k) \) possible outcomes, the expected utility expands to:

\[
\mathbb{E}_{p^{J}}\left[ U\left( BF_{\text{bettor}}(t+1) \right) \right] = \sum_{i_1=1}^{N(1)} \sum_{i_2=1}^{N(2)} \dots \sum_{i_M=1}^{N(M)} U\left( BF_{\text{bettor}}^{(i_1, i_2, \dots, i_M)}(t+1) \right) \times \prod_{k=1}^{M} p_{i_k}^{k,J}
\]

Where:

\begin{itemize}
    \item \( i_k \) indexes the outcome of match \( k \).
    \item \( BF_{\text{bettor}}^{(i_1, i_2, \dots, i_M)}(t+1) \) is the bankroll factor after all matches, given the outcomes \( i_1, i_2, \dots, i_M \).
    \item \( p_{i_k}^{k,J} \) is the estimated probability of outcome \( i_k \) for match \( k \).
    \item The product \( \prod_{k=1}^{M} p_{i_k}^{k,J} \) represents the joint probability of the specific combination of outcomes, assuming independence between matches.
\end{itemize}

For each outcome combination \( (i_1, i_2, \dots, i_M) \), the bankroll factor is calculated as:

\[
BF_{\text{bettor}}^{(i_1, i_2, \dots, i_M)}(t+1) = BF_{\text{bettor}}(t) \times \left( 1 + \sum_{k=1}^{M} f_{k,o_{i_k}} \left( o_{k,o_{i_k}} - 1 \right) \right)
\]

Where:

\begin{itemize}
    \item \( f_{k,o_{i_k}} \) is the fraction of the bankroll wagered on outcome \( o_{i_k} \) in match \( k \).
    \item \( o_{k,o_{i_k}} \) is the odds offered by the bookmaker for outcome \( o_{i_k} \) in match \( k \).
\end{itemize}

An analytic simplification demonstration for the Kelly criteria, \(u = ln\), can be found at the end of this work \ref{appendix:analytical_solution_using_the_kelly_criterion}.

\subsubsection{Importance of Accurate Probability Estimates}

The bettor's decisions hinge on their estimated probabilities. Inaccurate estimates can lead to sub-optimal betting strategies:

\begin{itemize}
     \item\textbf{Overestimation} of probabilities may cause the bettor to wager too much, increasing the risk of significant losses.
    \item \textbf{Underestimation} may result in conservative wagers, leading to missed opportunities for profit.
\end{itemize}

By accurately estimating the probabilities, the bettor can better align their strategy with their utility function, optimizing the trade-off between expected return and risk.


\subsection{Expected Bankroll Factor}

\paragraph{}
The expected value \( \mathbb{E} \) of the bankroll factor \( BF \) corresponds to a simple utility function \( U(B) = B \), representing a risk-neutral perspective. This expected value is crucial in understanding the growth of wealth without considering risk preferences. An analytic form for this expectation can be derived straightforwardly.


\paragraph{}
The expected bankroll factor at time \( t+1 \) incorporates the bettor's actions and the estimated probabilities of outcomes. The evolution of the bankroll factor from time \( t \) to \( t+1 \) is given by:

\[
BF_{\text{bettor}}^J(t+1) = BF_{\text{bettor}}^J(t) \left[ 1 + \sum_{k=1}^{M} \sum_{i=1}^{N^k} f_i^{k,J}(t) \left( o_i^{k,B}(t) X_i^k - 1 \right) \right]
\]

Here, \( X_i^k \) is an indicator variable that equals 1 if outcome \( \omega_i^k \) occurs and 0 otherwise. The term inside the square brackets represents the return on the bettor's wagers during time \( t \).

\subsubsection{Calculating Expected Bankroll Factor}

To find the expected bankroll factor at time \( t+1 \), we take the expectation with respect to the bettor's estimated probabilities \( p_i^{k,J} \):

\[
\mathbb{E}_{p^{J}}\left[ BF_{\text{bettor}}^J(t+1) \right] = BF_{\text{bettor}}^J(t) \left[ 1 + \sum_{k=1}^{M} \sum_{i=1}^{N^k} f_i^{k,J}(t) \left( o_i^{k,B}(t) p_i^{k,J} - 1 \right) \right]
\]

This expression shows that the expected growth of the bettor's bankroll factor depends on:

\begin{itemize}
    \item The fraction of the bankroll wagered \( f_i^{k,J}(t) \).
    \item The odds offered \( o_i^{k,B}(t) \).
    \item The bettor's estimated probabilities \( p_i^{k,J} \) of the outcomes.
\end{itemize}



\subsection{Variance of the Bankroll Factor}

The variance of the bankroll factor provides insight into the risk or uncertainty associated with the bettor's strategy. A higher variance indicates greater risk, which may or may not be acceptable depending on the bettor's utility function.

\subsubsection{Calculating Variance for a Single Match}

For a single match \( m^k \), the variance of the bankroll factor component due to that match is:

\[
\text{Var}_{p^{J}}\left[ BF_{\text{bettor}}^{J,k}(t+1) \right] = \left( BF_{\text{bettor}}^J(t) \right)^2 \text{Var}\left[ \sum_{i=1}^{N^k} f_i^{k,J}(t) \left( o_i^{k,B}(t) X_i^k - 1 \right) \right]
\]

Within match \( m^k \), the outcomes are mutually exclusive and collectively exhaustive, so we account for the covariance between different outcomes.

The variance expands to:

\[
\begin{aligned}
\text{Var}_{p^{J}}\left[ BF_{\text{bettor}}^{J,k}(t+1) \right] = & \left( BF_{\text{bettor}}^J(t) \right)^2 \Bigg[ \sum_{i=1}^{N^k} \left( f_i^{k,J}(t) o_i^{k,B}(t) \right)^2 \text{Var}[X_i^k] 
& - 2 \sum_{i<j} f_i^{k,J}(t) o_i^{k,B}(t) f_j^{k,J}(t) o_j^{k,B}(t) \text{Cov}[X_i^k, X_j^k] \Bigg]
\end{aligned}
\]

Given that \( X_i^k \) is a Bernoulli random variable with success probability \( r_i^k \), the true probability of outcome \( \omega_i^k \):

\[
\text{Var}[X_i^k] = r_i^k (1 - r_i^k)
\]

However, the bettor does not know \( r_i^k \) and may use their estimated probability \( p_i^{k,J} \) in their calculations. Despite this, the true variance depends on \( r_i^k \), reflecting the inherent risk in the actual outcomes.

For the covariance between different outcomes:

\[
\text{Cov}[X_i^k, X_j^k] = -r_i^k r_j^k
\]

This negative covariance arises because only one outcome can occur in a match.

\subsubsection{Total Variance Across All Matches}

Assuming independence between different matches, the total variance of the bankroll factor is the sum over all matches:

\[
\text{Var}_{p^{J}}\left[ BF_{\text{bettor}}^J(t+1) \right] = \left( BF_{\text{bettor}}^J(t) \right)^2 \sum_{k=1}^{M} \Bigg[ \sum_{i=1}^{N^k} \left( f_i^{k,J}(t) o_i^{k,B}(t) \right)^2 r_i^k (1 - r_i^k) - 2 \sum_{i<j} f_i^{k,J}(t) o_i^{k,B}(t) f_j^{k,J}(t) o_j^{k,B}(t) r_i^k r_j^k \Bigg]
\]

\paragraph{Implications for Risk Management}

Understanding the variance of the bankroll factor helps the bettor manage risk. A higher variance indicates that the bankroll factor is more sensitive to the outcomes of the bets, which could lead to larger fluctuations in wealth.


\subsection{Comparison of Objectives: Bettor vs. Bookmaker}

\subsubsection{Bettor's Perspective}

The bettor observes the odds \( o_i^{k,B}(t) \) offered by the bookmaker and decides on the fractions \( f_i^{k,J}(t) \) of their bankroll to wager on each outcome \( \omega_i^k \). The bettor's optimization problem is to choose \( f_i^{k,J}(t) \) to maximize their expected utility, given their estimated probabilities \( p_i^{k,J} \).

\subsubsection{Bookmaker's Perspective}

The bookmaker sets the odds \( o_i^{k,B}(t) \) before knowing the exact fractions \( f_i^{k,J}(t) \) that bettors will wager. The bookmaker faces uncertainty regarding the bettors' actions and must estimate the aggregate fractions:

\[
F_i^k(t) = \sum_{J} f_i^{k,J}(t)
\]

across all bettors.

The bookmaker's optimization problem involves setting the odds \( o_i^{k,B}(t) \) to maximize their expected utility, considering their own estimated probabilities \( p_i^{k,B} \) and their expectations about bettors' wagering behavior.

\subsubsection{Asymmetry and Strategic Interaction}

This asymmetry creates a strategic interaction:

\begin{itemize}
    \item \textbf{Bettor's Advantage:} The bettor acts after observing the odds, optimizing their bets based on their own estimated probabilities and the offered odds.
    \item \textbf{Bookmaker's Challenge:} The bookmaker sets the odds without knowing the exact betting fractions but must anticipate bettors' reactions. They need to estimate \( F_i^k(t) \) to manage risk and ensure profitability.
\end{itemize}

If the aggregate fractions wagered by bettors are biased relative to the true probabilities, the bookmaker's optimization may lead to odds that create opportunities for bettors. This can happen if bettors do not optimize their bets uniformly or have varying probability estimates, giving an advantage to informed bettors even when their estimated probabilities are closer to the bookmaker's than to the true probabilities.



\section{Conclusion}

While this framework provides a solid structure for understanding the dynamics of the betting market, it comes with several limitations. First, the assumptions of independent match outcomes and a static information environment simplify the complexity of real-world dynamics, where outcomes may be correlated, and new information arrives continuously. Additionally, we do not optimize the timing of bets, which is a critical factor in real betting markets where odds fluctuate over time.

Moreover, the non-additivity of certain utility functions, such as logarithmic and exponential utilities, limits the general interpretation of long-term gain and risk. While maximizing utility in each time period offers insights into short-term decision-making, it does not fully capture the long-term wealth dynamics, especially under more realistic non-additive frameworks. This can affect the risk management strategies of both bettors and bookmakers, particularly when considering future opportunities and evolving market conditions.

In the next section, we will focus on the implementation of a system to apply this framework and evaluate it in practice, through both simulation and the integration of real-world data. This will allow us to test the framework's assumptions and explore the effects of relaxing some of these limitations.
